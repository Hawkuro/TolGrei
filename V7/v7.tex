\documentclass[a4]{article}

\usepackage[icelandic]{babel}
\usepackage[T1]{fontenc}
\usepackage{amsmath}
\usepackage{graphicx}
\usepackage{sidecap}
\usepackage[utf8]{inputenc}
\usepackage[left=1in,top=1in,right=1in,bottom=1in,nohead]{geometry}


\title{Töluleg Greining\\ Verkefni 7}
\date{\today{}}
\author{ 
  Bjarki Geir Benediktsson,\and
  Haukur Óskar Þorgeirsson,\and
  Matthías Páll Gissurarson \and
  Kennari: Máni Maríus Viðarsson
  }
\begin{document}
\maketitle
\section{Dæmi 7}

Við notuðum það sem gefið var og fengum út eftirfarandi
mismunakvótatöflu.
$$
  \begin{array}{|l|l|l|l|l|l|l|}
    \hline
    i & x_i & f[x_i] & f[x_i,x_{i+1}] & f[x_i,x_{i+1},x_{i+2}] & f[x_i,x_{i+1},x_{i+2},x_{i+3}] & f[x_i,x_{i+1},x_{i+2},x_{i+3},x_i{+4}] \\
    \hline
    0 & -1 & 0.5 & 0 & 0.5 & -0.3125 & 0.125 \\
    1 & -1 & 0.5 & 0.5 & -0.125 & -0.0625 & \\
    2 & 0 & 1 & 0.25 & -0.25 & & \\
    3 & 1 & 1.25 & 0 & & & \\
    4 & 1 & 1.25 & & & &
  \end{array}
$$

en út frá henni fæst að

$$p(x) = 0.5 + 0.5(x+1)^2 - (5/16)(x+1)^2x + (1/8)(x+1)^2x(x-1)$$\\
$$= \frac{x^4}{8}-\frac{3x^3}{16}-0.25 x^2+ 0.5625 x+1,$$

og $$p(0.3) = 0.5 + 0.5\cdot 1.3^2 - (5/16)(1.3^2)\cdot 0.3 + 0.125
\cdot 1.3^2\cdot 0.3 \cdot -0.7 = 1.422.$$

Við fáum svo út frá ójöfnunni $-1 \leq f^{(5)}(x) \leq 4$ að
$$-0.00207025 = -1 \cdot \frac{(0.3+1)^2\cdot 0.3 \cdot (0.3-1)^2}{5!}
\leq f(0.3) - p(0.3) \leq 4 \cdot \frac{(0.3+1)^2\cdot 0.3 \cdot
  (0.3-1)^2}{5!} = 0.008281,$$ en lengd bilsin er $|0.008281 -
0.00207025| = 0.00621075$.  Ef við notum miðpunkt bilsins til að nálga
$f(0.3)$ og rúnum af miðað við leng bilsins fáum við fáum við þá
$f(0.3) = 1.422+0.003105375 = 1.4251 \pm 0.0031$.

\section{Dæmi 8}

\end{document}
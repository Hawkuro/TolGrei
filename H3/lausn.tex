\documentclass[a4]{article}

\usepackage[T1]{fontenc}
\usepackage[icelandic]{babel}
\usepackage{amsmath}
\usepackage{graphicx}
\usepackage{sidecap}
\usepackage[utf8]{inputenc}
\usepackage[left=1in,top=1in,right=1in,bottom=1in,nohead]{geometry}
\usepackage[framed,numbered,autolinebreaks,useliterate]{../mcode}
\usepackage{amsfonts}
\usepackage{epstopdf}

\lstset{language=MATLAB}
\title{Töluleg Greining\\ Heimaverkefni 3}
\date{\today{}}
\author{ 
  Bjarki Geir Benediktsson,\and
  Haukur Óskar Þorgeirsson,\and
  Matthías Páll Gissurarson \and
  Kennari: Máni Maríus Viðarsson
  }



\begin{document}
\begin{flushright}
  Bjarki Geir Benediktsson,\\
  Haukur Óskar Þorgeirsson,\\
  Matthías Páll Gissurarson\\
\end{flushright}

\begin{center}
 \textsc{ \LARGE Töluleg Greining\\
  Heimaverkefni 3\\
  \today{}
  }
  \end{center}
\vfill

\maketitle
\section{Forsagnar- og Leiðréttingaraðferð}
\subsection{}
\lstinputlisting[inputencoding=utf8]{adams_pc5.m}
\subsection{}
\lstinputlisting[inputencoding=utf8]{adams_pc5plot.m}
\lstinputlisting[inputencoding=utf8]{adams_pc5test.m}
%hér á að bæta inn ferilteikningum af prófunarfallinu
\subsection{}
til þess að fá heildarskekkju minni en $10^{-4}$ þurfti $h=\frac{1}{50000}=2*10^{-5}$

\subsection{Tímamælingar}
\lstinputlisting[inputencoding=utf8]{Timetake.m}
%athuga hvort það þurfi að breyta gildunum í take Taketime og uppfæra þá þessi gildi
Keyrsla á þessu forriti gaf eftirfarandi niðurstöður\\
\begin{tabular}{|c|c|}
\hline
fall		&meðatími \\\hline
adams\_pc5	&0,0082\\\hline
rkf45		&0,0024\\\hline
rkv56		&0,0027\\\hline
\end{tabular}


\section{Einfaldur pendúll}
Við fáum að þar sem $$\theta''(t) + \frac{g/l} \cdot \sin{ \theta(t)} \Leftrightarrow \theta''(t) = -\frac{g}{l} \cdot \sin{\theta(t)}$$
  þá má rita $$\left\{ \begin{array}{l} y(t) = \theta'(t) \\ y'(t) = \theta''(t) = - \frac{g}{l} \cdot \sin{\theta(t)} \end{array} \right.$$
Notum þetta til að skilgreina eftirfarandi fall:
\lstinputlisting[inputencoding=utf8]{pendulODE.m}
\lstinputlisting[inputencoding=utf8]{pendulApprox.m}

Við notuðum síðan þetta til að teikna hreyfimyndir, en það var gert með
\lstinputlisting[inputencoding=utf8]{pendull.m}

Með því að prófa nokkur gildi og horfa á útkomuna, þá komumst við að því að upphafhornið
$\theta_0 = 0.42$ gaf okkur frekar góða nálgun að 3 umferð, en eftir það fóru lausnirnar að greinast í sundur.

Til þess að svo gera hreyfimyndina þar sem útslagið er stórt var notaður eftirfarandi forritsbútur:
\lstinputlisting[inputencoding=utf8]{hr2.m}

\section{Róla}

\section{Kúlupendúll}

\section{Sólkerfi}
\vspace{20 mm}
Að skýrsluni unnu :
\hspace{0.5cm} \makebox[1.5in]{\hrulefill}
\hspace{0.5cm} \makebox[1.5in]{\hrulefill}
\hspace{0.5cm} \makebox[1.5in]{\hrulefill}
\end{document}

\documentclass[a4]{article}

\usepackage[icelandic]{babel}
\usepackage[T1]{fontenc}
\usepackage{amsmath}
\usepackage{graphicx}
\usepackage[utf8]{inputenc}
\usepackage[margin=3cm]{geometry}
\usepackage[framed,numbered,autolinebreaks,useliterate]{mcode}
\usepackage{amsfonts}
\usepackage{epstopdf}

\title{Töluleg Greining\\ Heimaverkefni 1}
\date{\today{}}
\author{ 
  Bjarki Geir Benediktsson,\and
  Haukur Óskar Þorgeirsson,\and
  Matthías Páll Gissurarson \and
  Kennari: Máni Maríus Viðarsson
  }



\begin{document}
\begin{flushright}
  Bjarki Geir Benediktsson,\\
  Haukur Óskar Þorgeirsson,\\
  Matthías Páll Gissurarson\\
\end{flushright}

\begin{center}
 \textsc{ \LARGE Töluleg Greining\\
  Heimaverkefni 1\\
  \today{}
  }
  \end{center}
\vfill

\maketitle
\section{Forsagnar- og Leiðréttingaraðferð}
\subsection{}
\lstinputlisting[inputencoding=utf8]{adams_pc5.m}
\subsection{}
%hér á að bæta inn ferilteikningum af prófunarfallinu
\subsection{}
til þess að fá heildarskekkju minni en $10^{-4}$ þurfti $h=\frac{1}{50000}=2*10^{-5}$

\subsection{Tímamælingar}
\lstinputlisting[inputencoding=utf8]{Timetake.m}
%athuga hvort það þurfi að breyta gildunum í take Taketime og uppfæra þá þessi gildi
Keyrsla á þessu forriti gaf eftirfarandi niðurstöður\\
\begin{tabular}{|c|c|}
\hline
fall		&meðatími \\\hline
adams\_pc5	&0,0082\\\hline
rkf45		&0,0024\\\hline
rkv56		&0,0027\\\hline
\end{tabular}


\section{Einfaldur pendúll}

\vspace{20 mm}
Að skýrsluni unnu :
\hspace{0.5cm} \makebox[1.5in]{\hrulefill}
\hspace{0.5cm} \makebox[1.5in]{\hrulefill}
\hspace{0.5cm} \makebox[1.5in]{\hrulefill}
\end{document}

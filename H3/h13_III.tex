\documentclass[11pt,icelandic]{article}
\usepackage[utf8]{inputenc}
\usepackage{mcode}
\usepackage{listingsutf8}
\usepackage{babel}
\usepackage{t1enc}
\usepackage{multicol}
\usepackage{amsmath}
\usepackage{amssymb}
\usepackage[dvips]{graphicx}      % usual driver
%\usepackage{pictex}
\selectlanguage{icelandic}
\columnsep=0.5truecm
\columnseprule=.01truecm
\hoffset=-2.2truecm
\voffset=-2truecm
\textwidth=16.5truecm 
\textheight=23truecm
\renewcommand{\Bbb}{\mathbb}
\newcommand{\C}{{\Bbb C}}
\newcommand{\R}{{\Bbb R}}
\newcommand{\D}{{\Bbb D}}
\newcommand{\M}{{\cal M}}
\newcommand{\I}{{\cal I}}
\newcommand{\F}{{\cal F}}
\renewcommand{\L}{{\cal L}}
\newcommand{\scalar}[2]{\langle#1,#2\rangle}
\newcommand{\norm}[1]{\|#1\|_{\varphi}}
\newcommand{\set}[1]{\{#1\}}
\newcommand{\adjoint}{\bar {\partial}^*_{\varphi}}
\renewcommand{\Re}{\operatorname{Re}}
\renewcommand{\Im}{\operatorname{Im}}
\newcommand{\Li}{\operatorname{Li}}
\newcommand{\Log}{\operatorname{Log}}
\newcommand{\stod}{\operatorname{supp}}
\newcommand{\nin}{\mbox{$\;\not\in\;$}}
\newcommand{\dive}{\mbox{${\rm\bf div\,}$}}
\newcommand{\curl}{\mbox{${\rm\bf curl\,}$}}
\newcommand{\grad}{\mbox{${\rm\bf grad\,}$}}
\newcommand{\spann}{\mbox{${\rm Span}$}}
\newcommand{\tr}{\mbox{${\rm tr}$}}
\newcommand{\rank}{\mbox{${\rm rank}$}}
\newcommand{\image}{\mbox{${\rm image}$}}
\newcommand{\nullity}{\mbox{${\rm null}$}}
\newcommand{\proj}{\mbox{${\rm proj}$}}
\newcommand{\id}{\mbox{${\rm id}$}}
\newcommand{\Av}{\mbox{${\bf A}$}}
\newcommand{\av}{\mbox{${\bf a}$}}
\newcommand{\uv}{\mbox{${\bf u}$}}
\newcommand{\vv}{\mbox{${\bf v}$}}
\newcommand{\wv}{\mbox{${\bf w}$}}
\newcommand{\xv}{\mbox{${\bf x}$}}
\newcommand{\zv}{\mbox{${\bf z}$}}
\newcommand{\yv}{\mbox{${\bf y}$}}
\newcommand{\bv}{\mbox{${\bf b}$}}
\newcommand{\cv}{\mbox{${\bf c}$}}
\newcommand{\dv}{\mbox{${\bf d}$}}
\newcommand{\ev}{\mbox{${\bf e}$}}
\newcommand{\fv}{\mbox{${\bf f}$}}
\newcommand{\gv}{\mbox{${\bf g}$}}
\newcommand{\hv}{\mbox{${\bf h}$}}
\newcommand{\iv}{\mbox{${\bf i}$}}
\newcommand{\jv}{\mbox{${\bf j}$}}
\newcommand{\kv}{\mbox{${\bf k}$}}
\newcommand{\pv}{\mbox{${\bf p}$}}
\newcommand{\nv}{\mbox{${\bf n}$}}
\newcommand{\qv}{\mbox{${\bf q}$}}
\newcommand{\rv}{\mbox{${\bf r}$}}
\newcommand{\sv}{\mbox{${\bf s}$}}
\newcommand{\tv}{\mbox{${\bf t}$}}
\newcommand{\ov}{\mbox{${\bf 0}$}}
\newcommand{\Fv}{\mbox{${\bf F}$}}
\newcommand{\Gv}{\mbox{${\bf G}$}}
\newcommand{\Nv}{\mbox{${\bf N}$}}
\newcommand{\Hv}{\mbox{${\bf H}$}}
\newcommand{\Ev}{\mbox{${\bf E}$}}
\newcommand{\Sv}{\mbox{${\bf S}$}}
\newcommand{\Tv}{\mbox{${\bf T}$}}
\newcommand{\Vv}{\mbox{${\bf V}$}}
\newcommand{\Bv}{\mbox{${\bf B}$}}
\newcommand{\Oa}{\mbox{$(0,0)$}}
\newcommand{\Ob}{\mbox{$(0,0,0)$}}
\newcommand{\Onv}{\mbox{$[0,0,\ldots,0]$}}
\newcommand{\an}{\mbox{$(a_1,a_2, \ldots,a_n)$}}
\newcommand{\xn}{\mbox{$(x_1,x_2, \ldots,x_n)$}}
\newcommand{\xnv}{\mbox{$[x_1,x_2, \ldots,x_n]$}}
\newcommand{\vnv}{\mbox{$[v_1,v_2, \ldots,v_n]$}}
\newcommand{\wnv}{\mbox{$[w_1,w_2, \ldots,w_n]$}}
\newcommand{\tvint}{\int\!\!\!\int}
\newcommand{\thrint}{\int\!\!\!\int\!\!\!\int}

%
\parindent 0pt
\thispagestyle{empty}
\begin{document}
\lstset{language=matlab}
%\lstset{inputencoding=latin1}  %Þetta er notað almennt. 
\lstset{inputencoding=utf8/latin1}
{\large\bf Háskóli Íslands}  \hfill {\large \bf Verkfræði- og
náttúruvísindasvið}

\bigskip
\begin{center}
{\Large\bf  Töluleg greining (STÆ405G) }
\end{center}

\begin{center}{\Large \bf  Heimaverkefni III \\ \medskip
Hreyfikerfi og hreyfimyndir
}
\end{center}

\begin{center}{\bf 22.3.2013}
\end{center}

\noindent
Tilgangur þessa verkefnis er að beita tölulegum aðferðum til þess að
leysa upphafsgildisverkefni og setja lausnir þeirra fram með hreyfimyndum.
Við tökum fyrir tvö stærðfræðileg líkön úr aflfræði (e.~dynamics) og 
síðan megið þið velja ykkur líkan eftir eigin höfði úr hvaða
hagnýtingu sem er, þó ekki einfaldan eða tvöfaldan pendúl, því þeir 
hafa mikið komið við sögu í þessu námskeiði á  undanförnum árum.  
Ef þið viljið taka fyrir pendúla, þá þurfa þeir að vera að minnsta
kosti þrefaldir.


\smallskip
Sígild aflfræði er geysilega ríkulegt og heillandi viðfangsefni.
Til eru  ógrynni af stærðfræðilegum líkönum úr raunvísindum og verkfræði þar
sem hægt er leysa með tölulegum aðferðum og setja fram með
hreyfimyndum.  Á Wikipedia er hægt að finna 
mjög fínar greinar um þessi efni og raunar um stærðfræði almennt.  
Ég hvet ykkur 
eindregið til þess að fletta upp ensku heitunum sem ég nefni í þessum 
texta á Wikipedia og lesa ykkur til um viðfangsefnin og leita að 
hugmyndum til fyrirmyndar að ykkar eigin verkefni.  Nemendur í
sveiflufræði geta til dæmis haft samráð við kennara sína.


\smallskip
Eftir að yfirferð á verkefninu er lokið þá munum við hafa sýningu á
vel völdum hreyfimyndum í fyrirlestri og veita verðlaun
sem bera yfirskriftina:

\smallskip

\begin{center}
 {\Large \it {\tt Matlab}-Eddan
fyrir bestu hreyfimyndina í tölulegri greiningu árið 2013}.   
\end{center}

\smallskip
Dómnefnd {\tt Matlab}-Eddunnar árið 2013 skipa þeir 
Guðmundur Einarsson, Jóhann Sigursteinn Björnsson og 
Máni Maríus Viðarsson.
Dómnefndin velur 10-20 myndir í úrslit en nemendur námskeiðsins 
kjósa bestu myndina.  Meir um þetta þegar nær dregur.




\section{Forsagnar-og-leiðréttingaraðferð} 


Nákvæmustu aðferðirnar sem sem fjallað er um í kennslubókinni okkar eru
\begin{itemize}
\item Fjórða-og-fimmta-stigs Runge-Kutta-Fehlberg-aðferð með
  breytilegri skrefstærð.
\item Fimmta-og-sjötta-stigs Runge-Kutta-Verner-aðferð með
  breytilegri skrefstærð.
\item Fjögurra-og-þriggja-skrefa Adams-forsagnar-og-leiðréttingaraðferð
\item Fjögurra-og-þriggja-skrefa Adams-forsagnar-og-leiðréttingaraðferð með
  breytilegri skrefstærð.
\end{itemize}

Á heimasíðu
höfundar kennslubókarinnar:
{\tt http://www.pcs.cnu.edu/$\sim$bbradie/mivps.html}  
finnið þið forritin  {\tt rk45, rkv56, adams{\_}pc4} og {\tt vs{\_}pc4.m}.
Afritið þau yfir á vélina ykkar.  Lesið vandlega grein 7.7 
á bls.~608-621 í kennslubók og glöggvið ykkur á uppbyggingu 
þessara forrita.

\vfill\eject
Þið eigið að forrita forsagnar-og-leiðréttingaraðferð þar 
sem forsagnarskrefið er tekið með fimm skrefa Adams-Bashforth-aðferð:
$$
\tilde w_j=w_{j-1}+h\big(
\tfrac{1901}{720} f(t_{j-1},w_{j-1})
-\tfrac{1387}{360} f(t_{j-2},w_{j-2})
+\tfrac{109}{30} f(t_{j-3},w_{j-3})
-\tfrac{637}{360} f(t_{j-4},w_{j-4})
+\tfrac{251}{720} f(t_{j-5},w_{j-5})
\big)
$$
og  leiðréttingarskrefið er tekið með 
fjögurra-skrefa Adams-Moulton-aðferð
$$
w_j=w_{j-1}+h\big(
\tfrac{251}{720} f(t_{j},\tilde w_{j})
+\tfrac{646}{720} f(t_{j-1},w_{j-1})
-\tfrac{264}{720} f(t_{j-2},w_{j-2})
+\tfrac{106}{720} f(t_{j-3},w_{j-3})
-\tfrac{19}{720} f(t_{j-4},w_{j-4})
\big).
$$
Báðar þessar aðferðir hafa 5.~stigs staðarskekkju.
Til þess að koma aðferðinni af stað þurfum við því að nota
5.~stigs Runge-Kutta-Fehlberg aðferð.  Þessari aðferð er lýst
með næst síðustu formúlunni á bls.~611 í kennslubókinni og 
$k_1,\dots,k_6$ eru reiknaðir skv.~formúlum á bls.~612.

\begin{enumerate}
\item [(i)]  Forritið á að heita {\tt adams{\_}pc5} og við mælum með því að 
það sé uppbyggt eins og  {\tt adams{\_}pc4} á heimasíðu
kennslubókarinnar {\tt http://www.pcs.cnu.edu/~bbradie/mivps.html} og
skilgreiningin á því á að vera:
{\tt 
function [wi, ti] = adams{\_}pc5 ( RHS, t0, x0, tf, N )
}

\item [(ii)]  Skrifið lýsingu á nálgunaraðferðinni fyrir framan
forritið í skýrslunni.  Lýsinguna 
á forritinu í {\tt m}-skránni megið þið hafa á ensku, eins
og í {\tt adams{\_}pc4}, en munið að aðlaga textann nýja forritinu.
\item [(iii)] Prófið forritið með því að finna nálgunarlausn á
$x'(t)=-3tx(t)^2+1/(1+t^3)$ með  $x(0)=0$, sem hefur réttu lausnina
$x(t)=t/(1+t^3)$.  (Sjá upphafið að grein 7.7 í kennslubók.) 
\item [(iv)] Teiknið fjórar myndir í skýrslunni tvær sem 
sýna upp réttu lausnina og nálgunarlausn fyrir sitt hvort gildið
á $h$, þannig að mismunur sjáist og tvær myndir með tilsvarandi
skekkju.  (Sjá myndir í grein 7.7) 
\item[(iv)] Finnið út hvað gildið á $h$ þarf að velja lítið 
til þess að  heildarskekkjan á bilinu $[0,5]$ sé minni en $10^{-4}$.
\item[(v)] Notið {\tt Matlab}-skipanirnar {\tt tic} og {\tt toc}
til þess að mæla hraðvirkni forritsins í samanburði við 
{\tt rk45} og {\tt rkv56}.  Gerið þetta með því að taka fína skiptingu
á bilinu $[0,5]$ og mæla hversu langan tíma það tekur fyrir hvert
forrit að ljúka verkefninu.  Gerið grein fyrir því hversu mörg
fallgildi $\fv(t,\wv)$ forritin þurfa að reikna út í hverju tilfelli.
Hvaða forrit er best?  
\end{enumerate}


\section{Einfaldur pendúll}


{\it Einfaldur pendúll} (e.~simple pendulum) í plani 
er heillandi fyrirbæri svo ekki sé sagt dáleiðandi. 
Hreyfijafnan fyrir útslagshorn pendúls er 
annars stigs ólínuleg jafna $\theta''+(g/\ell)\sin \theta=0$
þar sem  $g$ táknar þyngdarhröðunina og $\ell$ táknar
lengd pendúlsins.    Við köllum þetta einfaldan pendúl.

\smallskip
Ef útslagshornið er lítið þá er $\sin \theta\approx \theta$ 
og menn taka nálgun sem lausn línulegu afleiðujöfnunnar 
$\theta''+(g/\ell)\theta=0$ almenn lausn hennar er
af gerðinni $\theta(t)=A\cos(\omega\, t) + B\sin(\omega\, t)$
þar sem $\omega=\sqrt{g/\ell}$.  Þessi lausn er kölluð {\it harmónískur
  pendúll}.  Ef við setjum upphafsskilyrðin
$\theta(t_0)=\theta_0$ og 
$\theta'(t_0)=\theta_1$, þá fáum við lausnina
$$
\theta(t)=\theta_0\cos(\omega(t-t_0)) + \dfrac{\theta_1}\omega 
\sin(\omega(t-t_0)).
$$

Í möppunni {\it Forrit} á heimasíðu námskeiðsins á Uglunni finnið
þið skrána {\tt pendull.m}.  Í henni er sýnt hvernig hreyfimynd er
búin til fyrir harmóníska pendúlinn. Úttakið er 
skráin {\tt pendull.avi}.  Hana má síða spila í myndbandsforriti.


\smallskip
Fyrsta verk okkar er að hlaða niður þessu forriti og fá það til þess
að virka í {\tt Matlab}.  Gerið nokkrar tilraunir með forritið með því
að  breyta upphafsskilyrðunum.  


\smallskip
Takið eftir því að myndin er tvískipt.  Neðri parturinn sýnir sveiflu
pendúlsins, en efri parturinn er {\it fasarit} (e.~phase plot) 
jöfnunnar.  Það er almennt skilgreint fyrir lausn
$\xv(t)=[x_1(t),x_2(t)]^T$ á fyrsta 
stigs tvívíðu jöfnuhneppi  $\xv'(t)=\fv(t,\xv(t))$ sem braut 
lausnarinnar, þ.e.a.s.~ferillinn $t\mapsto (x_1(t),x_2(t))$ í 
$x_1x_2$-planinu.  Athugið að við teiknum fyrst allan ferilinn og
sýnum síðan á hreyfimyndinni hver staða pendúlsins er miðað við
fasaritið með $x_1(t)=\theta(t)$ og $x_2(t)=\theta'(t)$. 




\begin{enumerate}
\item [(i)]  Skrifið  upp fyrsta stigs hneppi 
sem er jafngilt jöfnu  $\theta''(t)+(g/\ell)\sin\theta(t)$
fyrir einfaldan pendúl.
\item [(ii)]  
Notið forritið {\tt adams{\_}pc5} sem þið gerðuð í síðasta lið
til þess að reikna út nálgun á hneppinu.
\item [(iii)]
Takið hreyfimyndina sem gefin er og búið nýja hreyfimynd 
þar sem harmóníski og einfaldi (ólínulegi) pendúlinn og 
tilsvarandi fasarit eru sýnd í ólíkum litum, en ferlarnir 
tveir eiga að vera  reiknaðir út með 
sömu upphafsskilyrðum.  Setjið fasaritin í tvær myndir
ofan fyrir ofan pendúlana.  Athugið að  þið getið þurft 
hafa  skrefstærðina í útreikningunum mjög litla til þess 
að fá góða nálgun, en þið þurfið um það bil 25 tímapunkta 
í einni lotu pendúlsins til þess að fá eðlilega hreyfimynd.  
Til viðmiðunar er lota harmóníska pendúlsins $T=2\pi\sqrt{\ell/g}$.
\item [(v)]  
Veljið nokkur gildi á upphafshorninu og metið fyrir
hvaða gildi á $\theta_0$ línulega nálgunin er góð í þrjár lotur og
hvenær lausnirnr fara að greinast í sundur.
 \item [(vi)]  Stækkið útslagið og búið til eina hreyfimynd þar sem
útslagið er stórt og mikið skilur á milli lausnanna tveggja.
Nefnið hreyfimyndina {\tt hr2.avi}  og
skilið henni með keyrsluskrá {\tt hr2.m} á Uglu.
\end{enumerate}



\section{Róla --  örvun með lotubundnum breytum}

Nú ætlum við að skoða fyrirbæri sem 
nefnt er á ensku  {\it parametric excitation}, {\it parametric
  resonance} og  {\it parametric pumping} og ég veit varla hvað heitir
á íslensku.  Köllum þetta {\it örvun með lotubundnum breytum}.  
Við þekkjum þetta fyrirbæri öll frá því að við vorum lítil og lékum okkur í
rólu. Þá komum við okkur fyrst á ferð og sveifluðum síðan fótunum
fram og aftur í takti eða úr takti við sveiflu rólunnar til þess að 
herða eða hægja á sveiflu rólunnar. 

\smallskip
Pendúll er punktmassi $m$ sem hangir í massalausri stöng.   Ef við
horfum á okkur sjálf í rólunni sem pendúl, þá hækkum við  og
lækkum  massamiðju okkar á víxl þegar við sveiflum fótunum.  
Það er eins og að hafa lengd pendúlsins tímaháða
$\ell=s(t)$

\smallskip
Við köllum þetta örvaðan pendúl.  Jafna hans er
$$
\theta''+\dfrac g{s(t)}\theta=0
$$
og við gerum tilraunir með að föll af gerðinni
$$
s(t)=\ell+a\cos(\omega t+\varphi)
$$
þar sem þið eigið að velja útslagið $a$, horntíðnina 
$\omega$ og fasahliðrunina $\varphi$.  

\smallskip
Munið að horntíðni hreintóna pendúlsins er $\sqrt{g/\ell}$ og 
lota hans  $T=2\pi\sqrt{\ell/g}$.   
Lota einfalda (ólínulega) pendúlsins er gefin með formúlu
$$
T=2\pi \sqrt{\dfrac \ell g}\bigg(
1+\dfrac 1{16} \theta_0^2+\dfrac{11}{3072} \theta_0^4+\cdots\bigg)
$$
þar sem $\theta_0$ er hámarksútslag hans.  Þessar staðreyndir
geta hjálpað ykkur við að velja horntíðnina $\omega$ og fasahliðrunina
$\varphi$ í þessum hluta verkefnisins. 


\begin{enumerate}
\item[(i)]  Fyrsta stigs hneppi sem er jafngild 
hreyfijöfnu örvaða pendúlsins.
\item[(ii)] 
Notið sömu aðferð og í síðasta lið til þess að
leysa afleiðujöfnuna og  
setjið lausnina fram með hreyfimynd eins og 
í síðasta lið með einfalda pendúlnum og örvaða pendúlnum í ólíkum
litum í sömu hreyfimynd. Skilið hreyfimyndinni {\tt hr3.avi} með
skýrslunni og keyrsluskránni {\tt hr3.m} sem býr hana til.
\item[(iii)] Fyrir hvaða gildi á  $\omega$ og $\varphi$ 
örvast útslag pendúlsins mest?
\item[(iv)] Hvaða gildi á $\varphi$ notar maður til þess að hægja á
  rólunni?
\item[(v)] Túlkið eðlisfræðilega hvernig þetta gengur fyrir sig.
\end{enumerate}



\section{Kúlupendúll}

Byrjið á því að fletta upp orðinu {\it pendulum} á Wikipedia
og renna ykkur niður síðuna þar til þið sjáið kúlupendúl sveiflast 
undir með mynd af Foucault í bakgrunni.  Við ætlum að reikna 
út nálganir á hreyfijöfnum  kúlupendúls og teikna upp lausnina
á mynd eins of þessari.  Takið eftir ferlinum sem
ritaður er í botninn. Við æltum líka að teikna hann.  

\smallskip
Kúlupendúll sem sveiflast í þyngdarsviði hefur hreyfiorkuna
$$
T=\tfrac 12 m\ell^2\dot\theta^2+\tfrac 12 m\ell^2(\sin^2\theta)
 \dot \phi^2
$$
þar sem $m$ er massi pendúlsins, $\ell$ er lengd hans, $\theta=\theta(t)$ er
hornið sem hann myndar við lóðlínu sem fall af tíma $t$. $\dot\theta$ er 
tímaafleiða $\theta$, fallið $\phi$ er hornið sem pendúllinn myndar við
$xz$-planið sem fall af tíma og $\dot\phi$ er afleiða
$\phi$ með tilliti til tíma.

\smallskip
Stöðuorka pendúlsins í þyngdarsviði er 
$$
U=-mg\ell\cos \theta
$$
og er hún því sett $0$ þegar pendúllinn stendur hornrétt á lóðrétta
ásinn. 

\smallskip
Lagrange-fallið er $L=T-U$ og skriðþungarnir sem svara til 
breytanna $\theta$ og $\phi$ eru
\begin{equation*}
p_\theta=\dfrac{\partial L}{\partial\dot\theta}=m\ell^2\dot\theta,
\qquad \text{ og } \qquad 
p_\phi=\dfrac{\partial L}{\partial\dot\phi}
=m\ell^2(\sin^2 \theta) \, \dot\phi
\end{equation*}
Heildarorkan er
$$
H=T+U=\dfrac{p_\theta^2}{2m\ell^2}+
\dfrac{p_\phi^2}{2m\ell^2\sin^2\theta}-m\ell g\cos \theta.
$$
Samkvæmt Hamilton-aflfræði (e.~hamiltonian mechanics) 
eru hreyfijöfnurnar fjórar:
\begin{align*}
  \dot\theta&=\dfrac{\partial H}{\partial p_\theta}=\dfrac{p_\theta}{m\ell^2},\\
  \dot\phi&=\dfrac{\partial H}{\partial
    p_\phi}=\dfrac{p_\phi}{m\ell^2\sin^2\theta},\\ 
\dot p_\theta&=-\dfrac{\partial H}{\partial \theta}
=\dfrac{p_\phi^2\cos\theta}{m\ell^2\sin^3\theta}-mg\ell\sin \theta,\\
\dot p_\phi&=-\dfrac{\partial H}{\partial \phi}=0.
\end{align*}
Það er þægilegast að  innleiða breyturnar
$x_1=\theta$, $x_2=\phi$, $x_3=p_\theta$ og $x_4=p_\phi$.
                                                                                                                                                                              
\smallskip
\begin{enumerate}
\item [(i)]  Skrifið upp fyrsta stigs hneppið sem er jafngilt
hreyfijöfnum kúlupendúlsins.
\item [(ii)] Notið nú forritið ykkar til þess að leysa jöfnuna og búið
til hreyfimynd.  Þið þurfið að breyta teikniforritinu og teikna
þrívíddarmynd með {\tt Matlab}-skipuninni {\tt plot3} til þess að 
teikna í þremur víddum.  Ímyndið ykkur að það renni blek frá penúlnum
niður á gólfið og að hann riti feril.   Í myndinni eigið þið 
að teikna ferilinn sem þannig myndast í gólfið fyrir neðan pendúlinn 
eða í aðra mynd líkt og fasamyndirnar í fyrri liðum.
\item [(iii)]  Skilið inn hreyfimynd  í {\tt hr4.avi}
og tilheyrandi keyrsluskrá {\tt hr4.m}, ásamt annarri grafík. 
\end{enumerate}

\section{ Hreyfikerfi að eigin vali}

Notið eins  aðferðir og í síðustu tveimur greinum til þess að
leysa afleiðujöfnuhneppi sem lýsir einhverju kerfi að eigin vali.
Notið þá grafísku framsetningu á lausninni sem ykkur finnst best eiga
við. 
\begin{enumerate}
\item [(i)] Notið eigið forrit til útreikninganna, en ef ykkur þykir
það ganga treglega eða ekki gefa nógu nákvæmar lausnir á skikkanlegum
tíma þá skuluð þið nota Runge-Kutta-Verner-aðferðina í forritinu
{\tt rkv56}.  Hún er nákvæmasta aðferðin sem eigum völ á.
\item [(ii)]  Skilið inn hreyfimynd/grafík  í skránni {\tt hr5.avi}
tilheyrandi keyrsluskrá {\tt hr5.m}. 
\end{enumerate}







\section*{Atriði um frágang og framsetningu:}  Lesið vandlega
forsíðu Heimaverkefnis I.  
Við yfirferð á verkefnunum þá verður eftirfarandi listi hafður til hliðsjónar
og er dregið niður fyrir þau atriði á honum sem er ábótavant.

\begin{itemize}
 \item Þjappið {\tt .avi} skrám og ekki gleyma að skila inn 
hreyfimyndaskrám og viðeigandi keyrsluskrám með nöfnum
sem nefnd eru í skýrslunni.
 \item Nöfn á forsíðu efst til hægri og undirskriftir á síðustu síðu.
 \item Forritakeyrslur eru í skýrslunni.
 \item Kaflaheiti og kaflanúmer eru eins og í fyrirmælunum
 \item Forritunarkóði og forritakeyrslur auðgreindar, og í leturgerð
sem hentar (t.d.~\verb=verbatim= og M-code LaTeX Package: mcode.sty,
(er að finna í Matlab-möppunni á Uglu)).
 \item Útskýringar í haus forrita, \verb=>>help= \emph{forrit} á að gefa
lýsingu á forritinu.
 \item Inntök og úttök forrita eru þau sömu og í fyrirmælunum, og í sömu röð.
 \item Skráarheiti og fallaheiti eins og í fyrirmælunum.
 \item Kóði inndreginn (Ctrl+A, Ctrl+I).
 \item Kóði útskýrður
\end{itemize}



\smallskip
\textbf{\large Um skil:}  Úrlausn á að skila ekki síðar en 
 kl.~16.00 föstudaginn 12.~apríl 2013.


\bigskip
Gangi ykkur vel! \\
Ragnar. 

\end{document}


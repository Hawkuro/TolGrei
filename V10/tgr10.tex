\documentclass[10pt,a4paper]{article}

\usepackage[T1]{fontenc} 
\usepackage[icelandic]{babel} 
\usepackage[utf8]{inputenc} 
\usepackage{epsfig}
\usepackage{amsmath}
\usepackage{amsfonts}
\usepackage[margin=3cm]{geometry}
\usepackage{listings}
\usepackage{graphicx}
\usepackage[section]{placeins}
%\usepackage[framed,numbered,autolinebreaks,useliterate]{mcode}
\input{kvmacros}

\newcommand{\Nz}{\mathbb{N}_0}
\newcommand{\ilc}{\lstinline}
\renewcommand{\thesubsection}{\alph{subsection})}
\newcommand{\fig}[3]{\begin{figure}[h!]
\centering
\includegraphics#1{#2}
\caption{#3}
\end{figure}}

\begin{document}

\title{Töluleg Greining\\Vikublað 10}
\author{ 
  Bjarki Geir Benediktsson,\and
  Haukur Óskar Þorgeirsson,\and
  Matthías Páll Gissurarson \and
  Kennari: Máni Maríus Viðarsson
  }
\maketitle

\setcounter{section}{8}

\section{16. júní 2011}

\subsection{}

Höfum að $R(h) = \frac{f(a+h) + f(a-h) - 2f(a)}{h^2}$ og, vegna þess hvernig punktarnir sem gefnir eru dreifast, $h=0.2$ og $a=0$.\\

Setjum upp í töflu:

\begin{tabular}{l|lll}
$h$&$D(i,0)$&$D(i,1)$\\ \hline
0.2&\\
0.1&\\
\end{tabular}

\end{document}

\documentclass[10pt,a4paper]{article}

\usepackage[T1]{fontenc} 
\usepackage[icelandic]{babel} 
\usepackage[utf8]{inputenc} 
\usepackage{epsfig}
\usepackage{amsmath}
\usepackage{amsfonts}
\usepackage[margin=3cm]{geometry}
\usepackage{listings}
\usepackage{graphicx}
\usepackage[section]{placeins}
%\usepackage[framed,numbered,autolinebreaks,useliterate]{mcode}
\input{kvmacros}

\newcommand{\Nz}{\mathbb{N}_0}
\newcommand{\ilc}{\lstinline}
\renewcommand{\thesubsection}{\alph{subsection})}
\newcommand{\fig}[3]{\begin{figure}[h!]
\centering
\includegraphics#1{#2}
\caption{#3}
\end{figure}}

\begin{document}

\title{Töluleg Greining\\Vikublað 10}
\author{ 
  Bjarki Geir Benediktsson,\and
  Haukur Óskar Þorgeirsson,\and
  Matthías Páll Gissurarson \and
  Kennari: Máni Maríus Viðarsson
  }
\maketitle

\setcounter{section}{8}

\section{16. júní 2011}

\subsection{}

Höfum að $R(h) = \frac{f(a+h) + f(a-h) - 2f(a)}{h^2}$ og, vegna þess hvernig punktarnir sem gefnir eru dreifast, $h=0.2$ og $a=0$.\\

Setjum upp í töflu:

\begin{tabular}{l|ll}
$h$&$D(i,0)$&$D(i,1)$\\ \hline
0.2&9,178025\\
0.1&8,945900&8,868525\\
\end{tabular}\\

Fáum út úr þessu nálgunina $8,868525$, með skekkjumat $7,7375 \times 10^{-2}$

\subsection{}

Höfum nú 
\[R(i,0)=T(h_i)=\sum_{k=0}^{2^i-1}h(\frac{1}{2}f(x_k) + \frac{1}{2}f(x_{i+k})),\quad i=0,1,2\, , \quad h=\frac{(0.4)}{2^i},\quad x_k =  -0.2 + kh\]

\[
R(i,j)=R(i,j-1) + \frac{1}{4^j-1}(R(i,j-1)-R(i-1,j-1))
\]

Fáum því eftirfarandi töflu:

\begin{tabular}{l|lll}
$h$&$R(i,0)$&$R(i,1)$&$R(i,2)$\\ \hline
0.4&0.4734242\\
0.2&0.4367121&0.42447473\\
0.1&0.42730195&0.42416523&0.4241445967\\
\end{tabular}

Þannig að við fáum nálgunina 0.4241445966 með skekkjumatið $2.06 \times 10^{-5}$

\setcounter{section}{8}

\section{6. maí 2011}
\subsection{}
til að fá skekkju minni en 0.01 höfum við $$h^4=\frac{180*0.01}{\alpha},\ \alpha = \max_{ x \in [1,2]}f(x) $$
hlutbillengdin þarf þá að vera minni en $$\sqrt[4]{\frac{1.8\cdot 2}{\pi^3}}\approx 0.58$$
og þá er 0.5 gott val því það er stærsta talan lægri en 0.58 sem gengur upp í 1 
\subsection{}
\[R(i,0)=T(h_i)=\sum_{k=0}^{2^i-1}h(\frac{1}{2}f(x_k) + \frac{1}{2}f(x_{i+k})),\quad i=0,1,2\, , \quad h=\frac{(1)}{2^i},\quad x_k =  1 + kh\]

\[
R(i,j)=R(i,j-1) + \frac{1}{4^j-1}(R(i,j-1)-R(i-1,j-1))
\]

Fáum því eftirfarandi töflu:

\begin{tabular}{l|lll}
$h$&$R(i,0)$&$R(i,1)$&$R(i,2)$\\ \hline
1	&0\\
0.5	&-1.03033&-1.3737733\\
0.25	&-1.239&-1.308556666&-1.304208891\\
\end{tabular}
Þannig að við fáum nálgunina -1.304208891

\end{document}

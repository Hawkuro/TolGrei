\documentclass[a4]{article}

\usepackage[icelandic]{babel}
\usepackage[T1]{fontenc}
\usepackage{amsmath}
\usepackage{sidecap}
\usepackage[utf8]{inputenc}
\usepackage[left=1in,top=1in,right=1in,bottom=1in,nohead]{geometry}
\usepackage[framed,numbered,autolinebreaks,useliterate]{mcode}%?essi l?na er til þess að taka inn latec kóða undir \begin{lstlisting}
\usepackage{amsfonts}

\title{Heimaverkefni 1}
\begin{document}
\begin{flushright}
  Töluleg Greining, Heimaverkefni 1\\
  \today{}\\
  Bjarki Geir Benediktsson,\\
  Haukur Óskar Þorgeirsson,\\
  Matthías Páll Gissurarson
  \end{flushleft}

\section{}
\lstinputlisting[language=Matlab]{square.m}

\section{}

\lstinputlisting[language=Matlab]{square_check.m} 
\section{}
\lstinputlisting[language=Matlab]{newton_gradient.m}

\section{}
\subsection*{fallið f úr lið i}
\lstinputlisting[language=Matlab]{func.m}
\subsection*{forritið}

Athugið að mcode virðist ekki styðja íslenska stafi, en útprentað er
``Hápunktur í'', ``Lágpunktur í'' og ``Söðulpunktur í'', ef punktur finnst,
en ``Ekki er hægt að segja til um (x,y) ='' ef ekki er hægt að segja til um punktinn.\\

\lstinputlisting[language=Matlab]{manualCriticalPointSearch.m}
\subsection*{}
forrit sem skilgreinir
$$f(x)=\sum_{j=1}^N \alpha_j e^{-\frac{||x-q_j||^2}{\epsilon}$$
\lstinputlisting[language=Matlab]{func2.m}
\section{}
\lstinputlisting[language=Matlab]{automaticCriticalPointSearch.m}
\end{document}
